\chapter*{Abstract}
Thema dieser Arbeit war, eine Augmented Reality Anwendung zu erstellen, welche Türen erkennt und Objekte (z.B. ein Vordach) über diese projiziert. Die Projektion soll dabei perspektivisch korrekt sein. Sprich, steht der Anwender seitlich zur Tür, dann sieh er auch das Vordach von der Seite. Die Anwendung sollte möglichst plattformunabhängig sein. Aus diesem Grund sollten möglichst alle verwendeten Bibliotheken Teil des Projekts sein und spezifisch für dieses kompiliert werden. So soll die Installation erleichtert werden. Zur Umsetzung wurde die OpenCV Bibliothek verwendet, welche ein plattformübergreifendes Toolset für die Bildverarbeitung bietet. Die Darstellung der Ausgabe geschieht mittels OpenGL.