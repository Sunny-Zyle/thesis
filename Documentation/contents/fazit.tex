\chapter{Fazit}

Insgesamt sind wir mit unserer Arbeit zufrieden und konnten unser Hauptziel zum gr�ssten Teil erreichen. Wir k�nnen T�ren in einem Livebild erkennen und deren Position und Rotation bestimmen. Die Arbeit war f�r uns eine grosse Herausforderung, da wir beide vorher nur wenig bis gar keine Erfahrung auf den unterschiedlichen Gebieten dieses Projekts hatten.

Probleme gibt es noch bei Erkennung von seitlich betrachteten T�ren. Hier funktioniert die Erkennung noch nicht zuverl�ssig. Wir haben zu sp�t realisiert, dass wir f�r die perspektivische Erkennung zus�tzliche Verfahren h�tten evaluieren sollen, z.B. die Klassifizierung von Linien anhand der Tendenz zu einem Fluchtpunkt.

% TODO evtl. mit Resultat verschmelze?


Da wir haben mit unserer Arbeit in erster Linie eine Grundlage geschaffen haben, gibt es definitiv zus�tzliches Optimiers- und Erweiterungspotential.

Sinnvoll w�re eine Optimierung der Performance der Algorithmen, besonders im Bereich der T�rerkennung. Hier haben wir prim�r auf die Funktionalit�t konzentriert. 

Ein wichtiger n�chster Schritt wird sein, die Anwendung auf mobile Plattformen zu portieren. Durch die immer h�ufigere Verbreitung von Smartphones und die M�glichkeit, Informationen an Ort und Stelle abzurufen, oder im Falle von Augmented Reality direkt einzublenden, sind solche Ger�te eine gew�nschte Zielgruppe. 

Ein Ausblick in die Zukunft der Augmented Reality sieht ebenfalls vielversprechend aus. Grosses Aufsehen sorgte diesbez�glich vorallem Google, die mit ihrem Produkt Glass versuchen Augmented Reality Massentauglich zu machen. Besonders in Alltagssituation k�nnte Augmented Reality eine grosse Rolle einnehmen, beispielsweise durch Navigationshilfen in Autos, die n�tzliche Informationen direkt auf die Frontscheibe projizieren.

Aber auch f�r unseren Spezialfall der Augmented Reality kann man sich verschiedene Anwendungszwecke vorstellen. Das Einkaufen k�nnte interaktiver werden, indem Dinge wie M�bel als Modelle auf das eigene Smartphone geladen werden. Mittels einer Augmented Reality App k�nnen diese direkt in der aktuellen Umgebung von allen Seiten betrachtet werden. Dies k�nne zudem mit aufkommenden Techniken wie NFC (Near Field Communication) erweitert werden: Man stelle sich eine Einkaufstour durch ein M�belhaus seiner Wahl vor. Mittels NFC k�nnen 3D-Modelle von M�beln direkt auf das Smartphone geladen werden. Sp�ter kann man die ausgesuchten Gegenst�nde in der eigenen Wohnung virtuell platzieren und dadurch eine bessere Kaufentscheidung f�llen.

