\chapter{Fazit}

Insgesamt sind wir mit unserer Arbeit zufrieden und konnten unser Hauptziel zum grössten Teil erreichen. Wir können Türen in einem Livebild erkennen und deren Position und Rotation bestimmen. Die Arbeit war für uns eine grosse Herausforderung, da wir beide vorher nur wenig bis gar keine Erfahrung auf den unterschiedlichen Gebieten dieses Projekts hatten.

Probleme gibt es noch bei Erkennung von seitlich betrachteten Türen. Hier funktioniert die Erkennung noch nicht zuverlässig. Wir haben zu spät realisiert, dass wir für die perspektivische Erkennung zusützliche Verfahren hätten evaluieren sollen, z.B. die Klassifizierung von Linien anhand der Tendenz zu einem Fluchtpunkt.

% TODO evtl. mit Resultat verschmelze?


Da wir haben mit unserer Arbeit in erster Linie eine Grundlage geschaffen haben, gibt es definitiv zusätzliches Optimiers- und Erweiterungspotential.

Sinnvoll wäre eine Optimierung der Performance der Algorithmen, besonders im Bereich der Türerkennung. Hier haben wir primär auf die Funktionalität konzentriert. 

Ein wichtiger nächster Schritt wird sein, die Anwendung auf mobile Plattformen zu portieren. Durch die immer häufigere Verbreitung von Smartphones und die Möglichkeit, Informationen an Ort und Stelle abzurufen, oder im Falle von Augmented Reality direkt einzublenden, sind solche Geräte eine gewünschte Zielgruppe. 

Ein Ausblick in die Zukunft der Augmented Reality sieht ebenfalls vielversprechend aus. Grosses Aufsehen sorgte diesbezüglich vorallem Google, die mit ihrem Produkt Glass versuchen Augmented Reality Massentauglich zu machen. Besonders in Alltagssituation könnte Augmented Reality eine grosse Rolle einnehmen, beispielsweise durch Navigationshilfen in Autos, die nützliche Informationen direkt auf die Frontscheibe projizieren.

Aber auch für unseren Spezialfall der Augmented Reality kann man sich verschiedene Anwendungszwecke vorstellen. Das Einkaufen könnte interaktiver werden, indem Dinge wie Möbel als Modelle auf das eigene Smartphone geladen werden. Mittels einer Augmented Reality App können diese direkt in der aktuellen Umgebung von allen Seiten betrachtet werden. Dies könnte zudem mit aufkommenden Techniken wie NFC (Near Field Communication) erweitert werden: Man stelle sich eine Einkaufstour durch ein Möbelhaus seiner Wahl vor. Mittels NFC können 3D-Modelle von Möbeln direkt auf das Smartphone geladen werden. Später kann man die ausgesuchten Gegenstände in der eigenen Wohnung virtuell platzieren und dadurch eine bessere Kaufentscheidung füllen.

